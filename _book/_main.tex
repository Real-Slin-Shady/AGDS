% Options for packages loaded elsewhere
\PassOptionsToPackage{unicode}{hyperref}
\PassOptionsToPackage{hyphens}{url}
%
\documentclass[
]{book}
\usepackage{amsmath,amssymb}
\usepackage{lmodern}
\usepackage{iftex}
\ifPDFTeX
  \usepackage[T1]{fontenc}
  \usepackage[utf8]{inputenc}
  \usepackage{textcomp} % provide euro and other symbols
\else % if luatex or xetex
  \usepackage{unicode-math}
  \defaultfontfeatures{Scale=MatchLowercase}
  \defaultfontfeatures[\rmfamily]{Ligatures=TeX,Scale=1}
\fi
% Use upquote if available, for straight quotes in verbatim environments
\IfFileExists{upquote.sty}{\usepackage{upquote}}{}
\IfFileExists{microtype.sty}{% use microtype if available
  \usepackage[]{microtype}
  \UseMicrotypeSet[protrusion]{basicmath} % disable protrusion for tt fonts
}{}
\makeatletter
\@ifundefined{KOMAClassName}{% if non-KOMA class
  \IfFileExists{parskip.sty}{%
    \usepackage{parskip}
  }{% else
    \setlength{\parindent}{0pt}
    \setlength{\parskip}{6pt plus 2pt minus 1pt}}
}{% if KOMA class
  \KOMAoptions{parskip=half}}
\makeatother
\usepackage{xcolor}
\usepackage{longtable,booktabs,array}
\usepackage{calc} % for calculating minipage widths
% Correct order of tables after \paragraph or \subparagraph
\usepackage{etoolbox}
\makeatletter
\patchcmd\longtable{\par}{\if@noskipsec\mbox{}\fi\par}{}{}
\makeatother
% Allow footnotes in longtable head/foot
\IfFileExists{footnotehyper.sty}{\usepackage{footnotehyper}}{\usepackage{footnote}}
\makesavenoteenv{longtable}
\usepackage{graphicx}
\makeatletter
\def\maxwidth{\ifdim\Gin@nat@width>\linewidth\linewidth\else\Gin@nat@width\fi}
\def\maxheight{\ifdim\Gin@nat@height>\textheight\textheight\else\Gin@nat@height\fi}
\makeatother
% Scale images if necessary, so that they will not overflow the page
% margins by default, and it is still possible to overwrite the defaults
% using explicit options in \includegraphics[width, height, ...]{}
\setkeys{Gin}{width=\maxwidth,height=\maxheight,keepaspectratio}
% Set default figure placement to htbp
\makeatletter
\def\fps@figure{htbp}
\makeatother
\setlength{\emergencystretch}{3em} % prevent overfull lines
\providecommand{\tightlist}{%
  \setlength{\itemsep}{0pt}\setlength{\parskip}{0pt}}
\setcounter{secnumdepth}{5}
\usepackage{booktabs}
\ifLuaTeX
  \usepackage{selnolig}  % disable illegal ligatures
\fi
\usepackage[]{natbib}
\bibliographystyle{plainnat}
\IfFileExists{bookmark.sty}{\usepackage{bookmark}}{\usepackage{hyperref}}
\IfFileExists{xurl.sty}{\usepackage{xurl}}{} % add URL line breaks if available
\urlstyle{same} % disable monospaced font for URLs
\hypersetup{
  pdftitle={Applied Geodata Science},
  pdfauthor={Benjamin Stocker (lead), Koen Hufkens (contributing), Pepa Aran (contributing), Pascal Schneider (contributing)},
  hidelinks,
  pdfcreator={LaTeX via pandoc}}

\title{Applied Geodata Science}
\author{Benjamin Stocker (lead), Koen Hufkens (contributing), Pepa Aran (contributing), Pascal Schneider (contributing)}
\date{2022-12-06}

\begin{document}
\maketitle

{
\setcounter{tocdepth}{1}
\tableofcontents
}
\hypertarget{about-this-book}{%
\chapter*{About this book}\label{about-this-book}}
\addcontentsline{toc}{chapter}{About this book}

This book accompanies the course(s) \emph{Applied Geodata Science}, taught at the Institute of Geography, University of Bern.

The course introduces the typical data science workflow using various examples of geographical and environmental data. With a strong hands-on component and a series of input lectures, the course introduces the basic concepts of data science and teaches how to conduct each step of the data science workflow. This includes the handling of various data formats, the formulation and fitting of robust statistical models, including basic machine learning algorithms, the effective visualisation and communication of results, and the implementation of reproducible workflows, founded in Open Science principles. The overall course goal is to teach students to tell a story with data.

\hypertarget{course-plan}{%
\chapter*{Course plan}\label{course-plan}}
\addcontentsline{toc}{chapter}{Course plan}

\begin{enumerate}
\def\labelenumi{\arabic{enumi}.}
\item
  Getting started
\item
  Programming primer
\item
  Data wrangling
\item
  Data visualisation
\item
  Data variety
\item
  Code management
\item
  Open Science practice

  \textbf{MILESTONE 1: Communicating a reproducible workflow (→ LO1)}
\item
  Regression
\item
  Supervised machine learning fundamentals
\item
  Random Forest
\item
  Neural Networks
\item
  Interpretable machine learning
\item
  Unsupervised machine learning

  \textbf{MILESTONE 2: Identify patterns and demonstrate how explained (→ LO2)}
\end{enumerate}

\hypertarget{getting_started}{%
\chapter{Getting started}\label{getting_started}}

\textbf{Chapter lead author: Pepa Aran}

TBC

Contents:

\begin{itemize}
\tightlist
\item
  Lecture (Beni): Data revolution, opportunities, challenges; explain relevance and why new methods are required
\item
  installing environment
\item
  workspace management
\item
  R, RStudio
\item
  R libraries, other libraries and applications
\end{itemize}

\hypertarget{learning-objectives}{%
\section{Learning objectives}\label{learning-objectives}}

\hypertarget{tutorial}{%
\section{Tutorial}\label{tutorial}}

\hypertarget{exercises}{%
\section{Exercises}\label{exercises}}

\hypertarget{solutions}{%
\section{Solutions}\label{solutions}}

\hypertarget{programming_primers}{%
\chapter{Programming primers}\label{programming_primers}}

\textbf{Chapter lead author: Pepa Aran}

TBC

Contents:

\begin{itemize}
\tightlist
\item
  Lecture (Beni): Models and data
\item
  Base R
\item
  variables, classes
\item
  data frames
\item
  loops
\item
  conditional statements
\item
  functions
\item
  input and output
\item
  intro to visualisation
\item
  Performance assessment: \href{https://stineb.netlify.app/files/ex1.pdf}{link} to my exercise, \href{https://github.com/stineb/EF_Activities/blob/master/Exercise_01_RPrimer.Rmd}{link to Dietze exercise}
\end{itemize}

\hypertarget{learning-objectives-1}{%
\section{Learning objectives}\label{learning-objectives-1}}

\hypertarget{tutorial-1}{%
\section{Tutorial}\label{tutorial-1}}

\hypertarget{exercises-1}{%
\section{Exercises}\label{exercises-1}}

\hypertarget{solutions-1}{%
\section{Solutions}\label{solutions-1}}

\hypertarget{data_wrangling}{%
\chapter{Data wrangling}\label{data_wrangling}}

\textbf{Chapter lead author: Benjamin Stocker}

Contents:

\begin{itemize}
\tightlist
\item
  Lecture (Beni): Tidy data, ``bad'' data
\item
  Data frame manipulations with tidyverse
\item
  Tidy data
\item
  Dealing with missingness, bad data, outliers
\item
  Imputation (note also imputation as part of the modelling workflow)
\item
  Performance assessment: \textbf{CAT 1,} \href{https://stineb.github.io/esds_book/ch-02.html\#exercise-1}{link}, Make table tidy
\end{itemize}

\hypertarget{learning-objectives-2}{%
\section{Learning objectives}\label{learning-objectives-2}}

\hypertarget{tutorial-2}{%
\section{Tutorial}\label{tutorial-2}}

\hypertarget{exercises-2}{%
\section{Exercises}\label{exercises-2}}

\hypertarget{solutions-2}{%
\section{Solutions}\label{solutions-2}}

\hypertarget{data_vis}{%
\chapter{Data visualisation}\label{data_vis}}

\textbf{Chapter lead author: Benjamin Stocker}

Contents:

\begin{itemize}
\tightlist
\item
  Lecture (Isabelle Bentz?): The art of visualising data, grammar of graphics
\item
  Exercise: Develop decision tree for what type of visualisation to apply
\item
  Performance assessment: Interactive work sequence
\end{itemize}

\hypertarget{learning-objectives-3}{%
\section{Learning objectives}\label{learning-objectives-3}}

\hypertarget{tutorial-3}{%
\section{Tutorial}\label{tutorial-3}}

\hypertarget{exercises-3}{%
\section{Exercises}\label{exercises-3}}

\hypertarget{solutions-3}{%
\section{Solutions}\label{solutions-3}}

\hypertarget{data_variety}{%
\chapter{Data variety}\label{data_variety}}

\textbf{Chapter lead author: Koen Hufkens}

Contents:

\begin{itemize}
\tightlist
\item
  Lecture (Mirko): Mapping data
\item
  Data formats, standards, metadata
\item
  Geographic data
\item
  Scraping, wget
\item
  APIs
\end{itemize}

\hypertarget{learning-objectives-4}{%
\section{Learning objectives}\label{learning-objectives-4}}

\hypertarget{tutorial-4}{%
\section{Tutorial}\label{tutorial-4}}

\hypertarget{exercises-4}{%
\section{Exercises}\label{exercises-4}}

\hypertarget{solutions-4}{%
\section{Solutions}\label{solutions-4}}

\hypertarget{code_mgmt}{%
\chapter{Code management}\label{code_mgmt}}

\textbf{Chapter lead author: Koen Hufkens}

Contents:

\begin{itemize}
\tightlist
\item
  git: repositories, stage, commit, push, fork, pull request, fetch upstream
\item
  Performance assessment: \textbf{CAT 2}
\end{itemize}

\hypertarget{learning-objectives-5}{%
\section{Learning objectives}\label{learning-objectives-5}}

\hypertarget{tutorial-5}{%
\section{Tutorial}\label{tutorial-5}}

\hypertarget{exercises-5}{%
\section{Exercises}\label{exercises-5}}

\hypertarget{solutions-5}{%
\section{Solutions}\label{solutions-5}}

\hypertarget{open_science}{%
\chapter{Open science practices}\label{open_science}}

\textbf{Chapter lead author: Koen Hufkens}

Contents:

\begin{itemize}
\tightlist
\item
  Lecture (Koen): Open science - history, motivation, reproducibility crisis, current initiatives, overview of practices
\item
  Environmental data repositories
\item
  Methods to create visualised reproducible workflow
\item
  RMarkdown files
\item
  Performance assessment: \textbf{CAT 3}, \href{https://github.com/stineb/EF_Activities/blob/master/Exercise_04_PairCoding.Rmd}{link to Dietze exercise on pair coding}
\end{itemize}

\hypertarget{learning-objectives-6}{%
\section{Learning objectives}\label{learning-objectives-6}}

\hypertarget{tutorial-6}{%
\section{Tutorial}\label{tutorial-6}}

\hypertarget{exercises-6}{%
\section{Exercises}\label{exercises-6}}

\hypertarget{solutions-6}{%
\section{Solutions}\label{solutions-6}}

\hypertarget{regression}{%
\chapter{Regression}\label{regression}}

\textbf{Chapter lead author: Benjamin Stocker}

Contents:

\begin{itemize}
\tightlist
\item
  Linear regression
\item
  Regression metrics
\item
  Logistic regression
\item
  classification metrics
\item
  Comparing models (AIC, \ldots)
\item
  Feature selection, stepwise regression, multi-colinearity (\href{http://www.sthda.com/english/articles/39-regression-model-diagnostics/160-multicollinearity-essentials-and-vif-in-r/}{vif})
\item
  Performance assessment: Exercise for stepwise regression \href{https://stineb.github.io/esds_book/ch-08.html}{link}
\end{itemize}

\hypertarget{learning-objectives-7}{%
\section{Learning objectives}\label{learning-objectives-7}}

\hypertarget{tutorial-7}{%
\section{Tutorial}\label{tutorial-7}}

\hypertarget{exercises-7}{%
\section{Exercises}\label{exercises-7}}

\hypertarget{solutions-7}{%
\section{Solutions}\label{solutions-7}}

\hypertarget{supervised_ml}{%
\chapter{Supervised machine learning}\label{supervised_ml}}

\textbf{Chapter lead author: Benjamin Stocker}

\begin{itemize}
\tightlist
\item
  Lecture (Beni): Overfitting, training, and cross-validation (\href{https://stineb.github.io/ml4ec_workshop/introduction.html\#overfitting}{link})
\item
  K nearest neighbour models
\item
  Data splitting
\item
  Preprocessing, standardization, imputation, dimension reduction, as part of the model training workflow
\item
  formula notation, recipes, generic train()
\item
  Training and loss function
\item
  Hyperparameters
\item
  Resampling
\item
  Performance assessment: Exercise comparing performance on test set of linear regression and KNN with different hyperparameter choices (like \href{https://stineb.github.io/ml4ec_workshop/solutions.html}{this}), discuss link to overfitting example
\end{itemize}

\hypertarget{learning-objectives-8}{%
\section{Learning objectives}\label{learning-objectives-8}}

\hypertarget{tutorial-8}{%
\section{Tutorial}\label{tutorial-8}}

\hypertarget{exercises-8}{%
\section{Exercises}\label{exercises-8}}

\hypertarget{solutions-8}{%
\section{Solutions}\label{solutions-8}}

\hypertarget{random_forest}{%
\chapter{Random forest}\label{random_forest}}

\textbf{Chapter lead author: Benjamin Stocker}

Contents:

\begin{itemize}
\tightlist
\item
  Lecture (Beni): Wisdom of the crowds, from decision trees to random forests
\item
  Performance assessment: Competition for best-performing model, given training-testing split of data; others should be able to reproduce performance
\end{itemize}

\hypertarget{learning-objectives-9}{%
\section{Learning objectives}\label{learning-objectives-9}}

\hypertarget{tutorial-9}{%
\section{Tutorial}\label{tutorial-9}}

\hypertarget{exercises-9}{%
\section{Exercises}\label{exercises-9}}

\hypertarget{solutions-9}{%
\section{Solutions}\label{solutions-9}}

\hypertarget{neural_nets}{%
\chapter{Neural networks}\label{neural_nets}}

\textbf{Chapter lead author: Benjamin Stocker}

Contents:

\begin{itemize}
\tightlist
\item
  Lecture (Beni): General introduction
\item
  Performance assessment: Competition for best-performing model, given training-testing split of data; others should be able to reproduce performance
\end{itemize}

\hypertarget{learning-objectives-10}{%
\section{Learning objectives}\label{learning-objectives-10}}

\hypertarget{tutorial-10}{%
\section{Tutorial}\label{tutorial-10}}

\hypertarget{exercises-10}{%
\section{Exercises}\label{exercises-10}}

\hypertarget{solutions-10}{%
\section{Solutions}\label{solutions-10}}

\hypertarget{interpretable_ml}{%
\chapter{Interpretable machine learning}\label{interpretable_ml}}

\textbf{Chapter lead author: Benjamin Stocker}

Contents:

\begin{itemize}
\tightlist
\item
  Variable importance
\item
  Partial dependency
\item
  Performance assessment: Compare partial dependency to a given predictor, detected with RF and with NN.
\end{itemize}

\hypertarget{learning-objectives-11}{%
\section{Learning objectives}\label{learning-objectives-11}}

\hypertarget{tutorial-11}{%
\section{Tutorial}\label{tutorial-11}}

\hypertarget{exercises-11}{%
\section{Exercises}\label{exercises-11}}

\hypertarget{solutions-11}{%
\section{Solutions}\label{solutions-11}}

  \bibliography{book.bib,packages.bib}

\end{document}
