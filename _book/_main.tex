% Options for packages loaded elsewhere
\PassOptionsToPackage{unicode}{hyperref}
\PassOptionsToPackage{hyphens}{url}
%
\documentclass[
]{book}
\usepackage{amsmath,amssymb}
\usepackage{lmodern}
\usepackage{iftex}
\ifPDFTeX
  \usepackage[T1]{fontenc}
  \usepackage[utf8]{inputenc}
  \usepackage{textcomp} % provide euro and other symbols
\else % if luatex or xetex
  \usepackage{unicode-math}
  \defaultfontfeatures{Scale=MatchLowercase}
  \defaultfontfeatures[\rmfamily]{Ligatures=TeX,Scale=1}
\fi
% Use upquote if available, for straight quotes in verbatim environments
\IfFileExists{upquote.sty}{\usepackage{upquote}}{}
\IfFileExists{microtype.sty}{% use microtype if available
  \usepackage[]{microtype}
  \UseMicrotypeSet[protrusion]{basicmath} % disable protrusion for tt fonts
}{}
\makeatletter
\@ifundefined{KOMAClassName}{% if non-KOMA class
  \IfFileExists{parskip.sty}{%
    \usepackage{parskip}
  }{% else
    \setlength{\parindent}{0pt}
    \setlength{\parskip}{6pt plus 2pt minus 1pt}}
}{% if KOMA class
  \KOMAoptions{parskip=half}}
\makeatother
\usepackage{xcolor}
\usepackage{longtable,booktabs,array}
\usepackage{calc} % for calculating minipage widths
% Correct order of tables after \paragraph or \subparagraph
\usepackage{etoolbox}
\makeatletter
\patchcmd\longtable{\par}{\if@noskipsec\mbox{}\fi\par}{}{}
\makeatother
% Allow footnotes in longtable head/foot
\IfFileExists{footnotehyper.sty}{\usepackage{footnotehyper}}{\usepackage{footnote}}
\makesavenoteenv{longtable}
\usepackage{graphicx}
\makeatletter
\def\maxwidth{\ifdim\Gin@nat@width>\linewidth\linewidth\else\Gin@nat@width\fi}
\def\maxheight{\ifdim\Gin@nat@height>\textheight\textheight\else\Gin@nat@height\fi}
\makeatother
% Scale images if necessary, so that they will not overflow the page
% margins by default, and it is still possible to overwrite the defaults
% using explicit options in \includegraphics[width, height, ...]{}
\setkeys{Gin}{width=\maxwidth,height=\maxheight,keepaspectratio}
% Set default figure placement to htbp
\makeatletter
\def\fps@figure{htbp}
\makeatother
\setlength{\emergencystretch}{3em} % prevent overfull lines
\providecommand{\tightlist}{%
  \setlength{\itemsep}{0pt}\setlength{\parskip}{0pt}}
\setcounter{secnumdepth}{5}
\usepackage{booktabs}
\ifLuaTeX
  \usepackage{selnolig}  % disable illegal ligatures
\fi
\usepackage[]{natbib}
\bibliographystyle{plainnat}
\IfFileExists{bookmark.sty}{\usepackage{bookmark}}{\usepackage{hyperref}}
\IfFileExists{xurl.sty}{\usepackage{xurl}}{} % add URL line breaks if available
\urlstyle{same} % disable monospaced font for URLs
\hypersetup{
  pdftitle={Applied Geodata Science},
  pdfauthor={Benjamin Stocker (lead), Koen Hufkens (contributing), Pepa Aran (contributing)},
  hidelinks,
  pdfcreator={LaTeX via pandoc}}

\title{Applied Geodata Science}
\author{Benjamin Stocker (lead), Koen Hufkens (contributing), Pepa Aran (contributing)}
\date{2022-11-22}

\begin{document}
\maketitle

{
\setcounter{tocdepth}{1}
\tableofcontents
}
\hypertarget{about-this-book}{%
\chapter*{About this book}\label{about-this-book}}
\addcontentsline{toc}{chapter}{About this book}

This book accompanies the course(s) \emph{Applied Geodata Science}, taught at the Institute of Geography, University of Bern.

The course introduces the typical data science workflow using various examples of geographical and environmental data. With a strong hands-on component and a series of input lectures, the course introduces the basic concepts of data science and teaches how to conduct each step of the data science workflow. This includes the handling of various data formats, the formulation and fitting of robust statistical models, including basic machine learning algorithms, the effective visualisation and communication of results, and the implementation of reproducible workflows, founded in Open Science principles. The overall course goal is to teach students to tell a story with data.

\hypertarget{course-plan}{%
\chapter*{Course plan}\label{course-plan}}
\addcontentsline{toc}{chapter}{Course plan}

\begin{enumerate}
\def\labelenumi{\arabic{enumi}.}
\item
  Getting started
\item
  Programming primer
\item
  Data wrangling
\item
  Data visualisation
\item
  Data variety
\item
  Code management
\item
  Open Science practice

  \textbf{MILESTONE 1: Communicating a reproducible workflow (→ LO1)}
\item
  Regression
\item
  Supervised machine learning fundamentals
\item
  Random Forest
\item
  Neural Networks
\item
  Interpretable machine learning
\item
  Unsupervised machine learning

  \textbf{MILESTONE 2: Identify patterns and demonstrate how explained (→ LO2)}
\end{enumerate}

\hypertarget{getting_started}{%
\chapter{Getting started}\label{getting_started}}

\textbf{Chapter lead author: Pepa Aran}

TBC

Contents:

\begin{verbatim}
- Lecture (Beni): Data revolution, opportunities, challenges; explain relevance and why new methods are required
- installing environment
- workspace management
- R, RStudio
- R libraries, other libraries and applications
\end{verbatim}

\hypertarget{programming_primers}{%
\chapter{Programming primers}\label{programming_primers}}

\textbf{Chapter lead author: Pepa Aran}

TBC

Contents:

\begin{verbatim}
- Lecture (Beni): Models and data
- Base R
- variables, classes
- data frames
- loops
- conditional statements
- functions
- input and output
- intro to visualisation
- Performance assessment: [link](https://stineb.netlify.app/files/ex1.pdf) to my exercise, [link to Dietze exercise](https://github.com/stineb/EF_Activities/blob/master/Exercise_01_RPrimer.Rmd)
\end{verbatim}

\hypertarget{data_wrangling}{%
\chapter{Data wrangling}\label{data_wrangling}}

\textbf{Chapter lead author: Benjamin Stocker}

Contents:

\begin{verbatim}
- Lecture (Beni): Tidy data, “bad” data
- Data frame manipulations with tidyverse
- Tidy data
- Dealing with missingness, bad data, outliers
- Imputation (note also imputation as part of the modelling workflow)
- Performance assessment: **CAT 1,** [link](https://stineb.github.io/esds_book/ch-02.html#exercise-1), Make table tidy
\end{verbatim}

\hypertarget{data_vis}{%
\chapter{Data visualisation}\label{data_vis}}

\textbf{Chapter lead author: Benjamin Stocker}

Contents:

\begin{verbatim}
- Lecture (Isabelle Bentz?): The art of visualising data, grammar of graphics
- Exercise: Develop decision tree for what type of visualisation to apply
- Performance assessment: Interactive work sequence
\end{verbatim}

\hypertarget{data_variety}{%
\chapter{Data variety}\label{data_variety}}

\textbf{Chapter lead author: Koen Hufkens}

Contents:

\begin{verbatim}
- Lecture (Mirko): Mapping data
- Data formats, standards, metadata
- Geographic data
- Scraping, wget
- APIs
\end{verbatim}

\hypertarget{code_mgmt}{%
\chapter{Code management}\label{code_mgmt}}

\textbf{Chapter lead author: Koen Hufkens}

Contents:

\begin{verbatim}
- git: repositories, stage, commit, push, fork, pull request, fetch upstream
- Performance assessment: **CAT 2**
\end{verbatim}

\hypertarget{open_science}{%
\chapter{Open science practices}\label{open_science}}

\textbf{Chapter lead author: Koen Hufkens}

Contents:

\begin{verbatim}
- Lecture (Koen): Open science - history, motivation, reproducibility crisis, current initiatives, overview of practices
- Environmental data repositories
- Methods to create visualised reproducible workflow
- RMarkdown files
- Performance assessment: **CAT 3**, [link to Dietze exercise on pair coding](https://github.com/stineb/EF_Activities/blob/master/Exercise_04_PairCoding.Rmd)

**MILESTONE 1: Communicating a reproducible workflow (→ LO1)**
\end{verbatim}

  \bibliography{book.bib,packages.bib}

\end{document}
